\documentclass[]{article}
\usepackage{lmodern}
\usepackage{amssymb,amsmath}
\usepackage{ifxetex,ifluatex}
\usepackage{fixltx2e} % provides \textsubscript
\ifnum 0\ifxetex 1\fi\ifluatex 1\fi=0 % if pdftex
  \usepackage[T1]{fontenc}
  \usepackage[utf8]{inputenc}
\else % if luatex or xelatex
  \ifxetex
    \usepackage{mathspec}
  \else
    \usepackage{fontspec}
  \fi
  \defaultfontfeatures{Ligatures=TeX,Scale=MatchLowercase}
\fi
% use upquote if available, for straight quotes in verbatim environments
\IfFileExists{upquote.sty}{\usepackage{upquote}}{}
% use microtype if available
\IfFileExists{microtype.sty}{%
\usepackage{microtype}
\UseMicrotypeSet[protrusion]{basicmath} % disable protrusion for tt fonts
}{}
\usepackage[margin=1in]{geometry}
\usepackage{hyperref}
\hypersetup{unicode=true,
            pdftitle={Follow-up on R},
            pdfborder={0 0 0},
            breaklinks=true}
\urlstyle{same}  % don't use monospace font for urls
\usepackage{color}
\usepackage{fancyvrb}
\newcommand{\VerbBar}{|}
\newcommand{\VERB}{\Verb[commandchars=\\\{\}]}
\DefineVerbatimEnvironment{Highlighting}{Verbatim}{commandchars=\\\{\}}
% Add ',fontsize=\small' for more characters per line
\usepackage{framed}
\definecolor{shadecolor}{RGB}{248,248,248}
\newenvironment{Shaded}{\begin{snugshade}}{\end{snugshade}}
\newcommand{\AlertTok}[1]{\textcolor[rgb]{0.94,0.16,0.16}{#1}}
\newcommand{\AnnotationTok}[1]{\textcolor[rgb]{0.56,0.35,0.01}{\textbf{\textit{#1}}}}
\newcommand{\AttributeTok}[1]{\textcolor[rgb]{0.77,0.63,0.00}{#1}}
\newcommand{\BaseNTok}[1]{\textcolor[rgb]{0.00,0.00,0.81}{#1}}
\newcommand{\BuiltInTok}[1]{#1}
\newcommand{\CharTok}[1]{\textcolor[rgb]{0.31,0.60,0.02}{#1}}
\newcommand{\CommentTok}[1]{\textcolor[rgb]{0.56,0.35,0.01}{\textit{#1}}}
\newcommand{\CommentVarTok}[1]{\textcolor[rgb]{0.56,0.35,0.01}{\textbf{\textit{#1}}}}
\newcommand{\ConstantTok}[1]{\textcolor[rgb]{0.00,0.00,0.00}{#1}}
\newcommand{\ControlFlowTok}[1]{\textcolor[rgb]{0.13,0.29,0.53}{\textbf{#1}}}
\newcommand{\DataTypeTok}[1]{\textcolor[rgb]{0.13,0.29,0.53}{#1}}
\newcommand{\DecValTok}[1]{\textcolor[rgb]{0.00,0.00,0.81}{#1}}
\newcommand{\DocumentationTok}[1]{\textcolor[rgb]{0.56,0.35,0.01}{\textbf{\textit{#1}}}}
\newcommand{\ErrorTok}[1]{\textcolor[rgb]{0.64,0.00,0.00}{\textbf{#1}}}
\newcommand{\ExtensionTok}[1]{#1}
\newcommand{\FloatTok}[1]{\textcolor[rgb]{0.00,0.00,0.81}{#1}}
\newcommand{\FunctionTok}[1]{\textcolor[rgb]{0.00,0.00,0.00}{#1}}
\newcommand{\ImportTok}[1]{#1}
\newcommand{\InformationTok}[1]{\textcolor[rgb]{0.56,0.35,0.01}{\textbf{\textit{#1}}}}
\newcommand{\KeywordTok}[1]{\textcolor[rgb]{0.13,0.29,0.53}{\textbf{#1}}}
\newcommand{\NormalTok}[1]{#1}
\newcommand{\OperatorTok}[1]{\textcolor[rgb]{0.81,0.36,0.00}{\textbf{#1}}}
\newcommand{\OtherTok}[1]{\textcolor[rgb]{0.56,0.35,0.01}{#1}}
\newcommand{\PreprocessorTok}[1]{\textcolor[rgb]{0.56,0.35,0.01}{\textit{#1}}}
\newcommand{\RegionMarkerTok}[1]{#1}
\newcommand{\SpecialCharTok}[1]{\textcolor[rgb]{0.00,0.00,0.00}{#1}}
\newcommand{\SpecialStringTok}[1]{\textcolor[rgb]{0.31,0.60,0.02}{#1}}
\newcommand{\StringTok}[1]{\textcolor[rgb]{0.31,0.60,0.02}{#1}}
\newcommand{\VariableTok}[1]{\textcolor[rgb]{0.00,0.00,0.00}{#1}}
\newcommand{\VerbatimStringTok}[1]{\textcolor[rgb]{0.31,0.60,0.02}{#1}}
\newcommand{\WarningTok}[1]{\textcolor[rgb]{0.56,0.35,0.01}{\textbf{\textit{#1}}}}
\usepackage{longtable,booktabs}
\usepackage{graphicx,grffile}
\makeatletter
\def\maxwidth{\ifdim\Gin@nat@width>\linewidth\linewidth\else\Gin@nat@width\fi}
\def\maxheight{\ifdim\Gin@nat@height>\textheight\textheight\else\Gin@nat@height\fi}
\makeatother
% Scale images if necessary, so that they will not overflow the page
% margins by default, and it is still possible to overwrite the defaults
% using explicit options in \includegraphics[width, height, ...]{}
\setkeys{Gin}{width=\maxwidth,height=\maxheight,keepaspectratio}
\IfFileExists{parskip.sty}{%
\usepackage{parskip}
}{% else
\setlength{\parindent}{0pt}
\setlength{\parskip}{6pt plus 2pt minus 1pt}
}
\setlength{\emergencystretch}{3em}  % prevent overfull lines
\providecommand{\tightlist}{%
  \setlength{\itemsep}{0pt}\setlength{\parskip}{0pt}}
\setcounter{secnumdepth}{5}
% Redefines (sub)paragraphs to behave more like sections
\ifx\paragraph\undefined\else
\let\oldparagraph\paragraph
\renewcommand{\paragraph}[1]{\oldparagraph{#1}\mbox{}}
\fi
\ifx\subparagraph\undefined\else
\let\oldsubparagraph\subparagraph
\renewcommand{\subparagraph}[1]{\oldsubparagraph{#1}\mbox{}}
\fi

%%% Use protect on footnotes to avoid problems with footnotes in titles
\let\rmarkdownfootnote\footnote%
\def\footnote{\protect\rmarkdownfootnote}

%%% Change title format to be more compact
\usepackage{titling}

% Create subtitle command for use in maketitle
\providecommand{\subtitle}[1]{
  \posttitle{
    \begin{center}\large#1\end{center}
    }
}

\setlength{\droptitle}{-2em}

  \title{Follow-up on R}
    \pretitle{\vspace{\droptitle}\centering\huge}
  \posttitle{\par}
    \author{}
    \preauthor{}\postauthor{}
    \date{}
    \predate{}\postdate{}
  

\begin{document}
\maketitle

{
\setcounter{tocdepth}{2}
\tableofcontents
}
\hypertarget{about-this-course}{%
\section*{About this course}\label{about-this-course}}
\addcontentsline{toc}{section}{About this course}

\textbf{\emph{Course Content}}

This course builds on introductory R and expands on essential concepts for R. We cover common problems in data manipulation to pre-process datasets and we will cover some visualization. \(99.5\%\) of a data scientist's job is data pre-processing and visualization. The rest is analysis. Therefore, brushing up on programming skills in R makes the experience of working with R a lot more enjoyable.

\textbf{\emph{Course Objectives}}

Participants will learn data manipulation skills such as merging data, re-shaping it, aggregating it and more. Course participants should be a lot more comfortable in working with R upon completing this course.

\textbf{\emph{Course Prerequisites}}

Participants are expected to have used R and RStudio before.

\textbf{\emph{Agenda}}

\begin{enumerate}
\def\labelenumi{\arabic{enumi}.}
\tightlist
\item
  Data sub-setting and logical conditions
\item
  Tidyverse introduction: Simple data manipulation, renaming, summarizing, aggregating
\item
  Merging data
\item
  Reshaping data
\item
  Regular expressions
\end{enumerate}

\textbf{\emph{Acknowledgements}}

The infrastructure for this website is in large parts adopted from work by UCL's \href{https://iris.ucl.ac.uk/iris/browse/profile?upi=ALIAX58}{Altaf Ali}, \href{https://www.jackblumenau.com/}{Jack Blumenau}, \href{https://lucasleemann.ch}{Lucas Leemann}, \href{https://sjankin.com/}{Slava Jankin Mikhaylov}, and \href{https://philippbroniecki.com}{Philipp Broniecki}. The ESRC Business and Local Government Data Research Centre is funded by the \href{https://esrc.ukri.org}{Economic and Social Research Council (ESRC)}.

\begin{center}\rule{0.5\linewidth}{\linethickness}\end{center}

placeholder

placeholder

\hypertarget{data-sub-setting-and-logical-conditions}{%
\section{Data sub-setting and logical conditions}\label{data-sub-setting-and-logical-conditions}}

\hypertarget{learning-objectives}{%
\subsection{Learning objectives}\label{learning-objectives}}

ABC\ldots{}

\hypertarget{tidyverse-introduction}{%
\section{Tidyverse introduction}\label{tidyverse-introduction}}

\hypertarget{seminar}{%
\subsection{Seminar}\label{seminar}}

\hypertarget{introduction-to-the-tidyverse}{%
\subsubsection{Introduction to the tidyverse}\label{introduction-to-the-tidyverse}}

The \texttt{tidyverse} package makes it easier to pre-process data. By pre-processing, we mean storing data in such a way that observations are in rows and variables are in columns. Data comes in different forms and is not always stored in this way. We will use data from the World Bank to illustrate this point.

Before getting started, we first need to install the \texttt{tidyverse} package like so: \texttt{install.packages("tidyverse")}. You only need to install once. However, doing this again is not a mistake. In fact, R, RStudio and R packages are regularly updated. It is good practice to update all of these on your computer as well. Just remember never to update before a deadline!

\begin{Shaded}
\begin{Highlighting}[]
\CommentTok{# clear workspace}
\KeywordTok{rm}\NormalTok{(}\DataTypeTok{list =} \KeywordTok{ls}\NormalTok{())}

\CommentTok{# load tidyverse package}
\KeywordTok{library}\NormalTok{(tidyverse)}
\end{Highlighting}
\end{Shaded}

Let's check whether we have updates available by runinng \texttt{tidyverse\_update()}

\begin{Shaded}
\begin{Highlighting}[]
\CommentTok{# check for available updates}
\KeywordTok{tidyverse_update}\NormalTok{()}
\end{Highlighting}
\end{Shaded}

\begin{verbatim}
The following packages are out of date:
  
* httr   (1.4.0 -> 1.4.1)
* modelr (0.1.4 -> 0.1.5)
* tidyr  (0.8.3 -> 1.0.0)
* xml2   (1.2.0 -> 1.2.2)

Start a clean R session then run:
install.packages(c("httr", "modelr", "tidyr", "xml2"))
\end{verbatim}

R tells us that several packages are in indeed out of data (this may be different on your computer - it's possible that everything is up to date on your machine). Below, we update according to the console message.

\begin{Shaded}
\begin{Highlighting}[]
\KeywordTok{install.packages}\NormalTok{(}\KeywordTok{c}\NormalTok{(}\StringTok{"httr"}\NormalTok{, }\StringTok{"modelr"}\NormalTok{, }\StringTok{"tidyr"}\NormalTok{, }\StringTok{"xml2"}\NormalTok{))}
\KeywordTok{tidyverse_update}\NormalTok{()}
\end{Highlighting}
\end{Shaded}

\begin{verbatim}
All tidyverse packages up-to-date
\end{verbatim}

Now, everything is updated correctly.

Let's check the codebook of our data.

\begin{longtable}[]{@{}ll@{}}
\toprule
\begin{minipage}[b]{0.11\columnwidth}\raggedright
Variable Name\strut
\end{minipage} & \begin{minipage}[b]{0.83\columnwidth}\raggedright
Description\strut
\end{minipage}\tabularnewline
\midrule
\endhead
\begin{minipage}[t]{0.11\columnwidth}\raggedright
IMMBRIT\strut
\end{minipage} & \begin{minipage}[t]{0.83\columnwidth}\raggedright
Out of every 100 people in Britain, how many do you think are immigrants from Non-western countries?\strut
\end{minipage}\tabularnewline
\begin{minipage}[t]{0.11\columnwidth}\raggedright
over.estimate\strut
\end{minipage} & \begin{minipage}[t]{0.83\columnwidth}\raggedright
1 if estimate is higher than 10.7\%.\strut
\end{minipage}\tabularnewline
\begin{minipage}[t]{0.11\columnwidth}\raggedright
RSex\strut
\end{minipage} & \begin{minipage}[t]{0.83\columnwidth}\raggedright
1 = male, 2 = female\strut
\end{minipage}\tabularnewline
\begin{minipage}[t]{0.11\columnwidth}\raggedright
RAge\strut
\end{minipage} & \begin{minipage}[t]{0.83\columnwidth}\raggedright
Age of respondent\strut
\end{minipage}\tabularnewline
\begin{minipage}[t]{0.11\columnwidth}\raggedright
Househld\strut
\end{minipage} & \begin{minipage}[t]{0.83\columnwidth}\raggedright
Number of people living in respondent's household\strut
\end{minipage}\tabularnewline
\begin{minipage}[t]{0.11\columnwidth}\raggedright
party\_self\strut
\end{minipage} & \begin{minipage}[t]{0.83\columnwidth}\raggedright
1 = Conservatives; 2 = Labour; 3 = SNP; 4 = Ukip; 5 = BNP; 6 = GP; 7 = party.other\strut
\end{minipage}\tabularnewline
\begin{minipage}[t]{0.11\columnwidth}\raggedright
paper\strut
\end{minipage} & \begin{minipage}[t]{0.83\columnwidth}\raggedright
Do you normally read any daily morning newspaper 3+ times/week?\strut
\end{minipage}\tabularnewline
\begin{minipage}[t]{0.11\columnwidth}\raggedright
WWWhourspW\strut
\end{minipage} & \begin{minipage}[t]{0.83\columnwidth}\raggedright
How many hours WWW per week?\strut
\end{minipage}\tabularnewline
\begin{minipage}[t]{0.11\columnwidth}\raggedright
religious\strut
\end{minipage} & \begin{minipage}[t]{0.83\columnwidth}\raggedright
Do you regard yourself as belonging to any particular religion?\strut
\end{minipage}\tabularnewline
\begin{minipage}[t]{0.11\columnwidth}\raggedright
employMonths\strut
\end{minipage} & \begin{minipage}[t]{0.83\columnwidth}\raggedright
How many mnths w. present employer?\strut
\end{minipage}\tabularnewline
\begin{minipage}[t]{0.11\columnwidth}\raggedright
urban\strut
\end{minipage} & \begin{minipage}[t]{0.83\columnwidth}\raggedright
Population density, 4 categories (highest density is 4, lowest is 1)\strut
\end{minipage}\tabularnewline
\begin{minipage}[t]{0.11\columnwidth}\raggedright
health.good\strut
\end{minipage} & \begin{minipage}[t]{0.83\columnwidth}\raggedright
How is your health in general for someone of your age? (0: bad, 1: fair, 2: fairly good, 3: good)\strut
\end{minipage}\tabularnewline
\begin{minipage}[t]{0.11\columnwidth}\raggedright
HHInc\strut
\end{minipage} & \begin{minipage}[t]{0.83\columnwidth}\raggedright
Income bands for household, high number = high HH income\strut
\end{minipage}\tabularnewline
\bottomrule
\end{longtable}

The dataset is on your memory sticks and also available for download \href{http://philippbroniecki.github.io/ML2017.io/data/BSAS_manip.RData}{here}.

\begin{Shaded}
\begin{Highlighting}[]
\CommentTok{# load non-western foreigners data set}
\KeywordTok{load}\NormalTok{(}\StringTok{"non_western_immigrants.RData"}\NormalTok{)}
\end{Highlighting}
\end{Shaded}


\end{document}
